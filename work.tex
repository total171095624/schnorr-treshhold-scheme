\chapter{Подпись Шнорра}

\section{Схема разделения секрета Шамира}

В работе \cite{KG} Шамир ставит перед собой задачу поделить секрет \textit{s} на \textit{n} частей так, чтобы 
\begin{itemize}
    \item знание любых \textit{k} или более частей $s_i$ позволяло легко вычислить s;
    \item знание любых $k-1$ или менее частей $s_i$ оставляло \textit{s} неопределённым в том смысле, что все его возможные значения равновероятны.
\end{itemize}

Полученный алгоритм называется (\textit{k}, \textit{n}) пороговой схемой. Полученная схема базируется на интерполяционных многочленах Лагранжа и выглядит она следующим образом:

Пусть $s \in Z_q$, $n < q$, тогда дилер выбирает случайный многочлен $f$ степени не более, чем $k-1$ на полем $Z_q$, такой, что $f(0) = s.$. Тогда каждый игрок в качестве своего кусочка получает $s(i) = f(i)$.  

Существует ровно один многочлен степени не более, чем $k - 1$, удовлетворяющий $f(i) = s_i$ для \textit{k} значений \textit{i}. Таким образом любая коалиция $\mathcal{P}$ из \textit{k} человек может восстановить многочлен $f$ путём интерполяции Лагранжа:

$$ f(u) = \sum_{i \in \mathcal{P}}f(i)\omega_i(u), \text{ где } \omega_i(u) = \prod_{\substack{i \in P \\ j \neq i}} \frac{u - j}{i - j} \text{ mod q.} $$

Так как $s = f(0)$, коалиция $\mathcal{P}$ может восстановить секрет следующим образом:

$$ s = f(0) = \sum_{i\in\mathcal{P}}f(i)\omega_i, \text{ где } \omega_i = \omega_i(0) = \prod_{\substack{i \in P \\ j \neq i}} \frac{j}{j - i} \text{ mod q.}$$

Каждое $\omega_i \neq 0$ и может быть легко вычислено из открытой информации. $k - 1$ участника для восстановления не хватит, так как иначе им не удастся ннайти свободный член многочлена $f$, потому что он с равной вероятностью может принимать любое из своих значений.

\section{Проверяемое разделение секрета Педерсена}

Схема проверяемого разделения секрета нужна, чтобы предотвратить жульничество со стороны дилера. В этой схеме каждый участник может проверить свою часть секрета. Если дилер поделился неверным секретом, это будет обнаружено. В работе \cite{VSS} Педерсен представил свой варинант подобной схемы. Выглядит она следующим образом.

Предположим, что дилер имеет секрет $s \in Z_q$ и случайное число $s' \in Z_q$, и определяется парой чисел (\textit{s}, \textit{s'}) через открытую информацию $C_0 = sG + s'H.$ Тогда секрет \textit{s} может быть разделён между пользователями так:

\begin{enumerate}
    \item Дилер выбирает случайные многочлены
    $$ f(u) = s + f_1u + ...+f_{t-1}u^{t-1} \text{ и } f'(u) = s' + f'_1u + ... + f'_{t-1}u^{t-1}$$
    где $s, s', f_t, f'_t \in Z_q$ для $t \in \{1, ... , k-1\}$. Подсчитывает ($s_i, s'_i$) = ($f(i), f'(i)$) для $i \in \{1, ..., n\}$.

    \item Диллер секретно рассылает участникам пары $(s_i, s'_i)$.

    \item Дилер предъявляет значения $C_t = f_tG + f'_tH$ for $t \in \{1, ..., k-1\}$.
\end{enumerate}

\begin{enumerate}
    \item Каждый участник проверяет, что 
    \begin{equation}\label{check}
        s_iG + s'_iH = \sum_{t=0}^{k-1}i^tC_t. 
    \end{equation}
     Если проверка провалилась, участник подаёт жалобу на дилера.
     \item От каждой жалобы от \textit{i}-го участника дилер может защитить себя предъявив значения $f(i), f'(i)$, удовлетворяющие (\ref{check}).
     \item Отказываемся от дилера, если
     \begin{itemize}
         \item На шаге 1 он получил хотя бы \textit{k} жалоб;
         \item На шаге 2 он ответил на жалобу значениями, не удовлетворяющими (\ref{check}).
     \end{itemize}

     Опираясь на вычислительную трудность проблемы дискретного логарифма на эллиптическй кривой, Педерсен доказал, что любая коалифция из менее, чем k участников не сможет получить никакой информации о секрете.
\end{enumerate}

\section{Генерация случайного секрета для криптосистем, опирающихся на проблему дискретного логарифма}

Приведём схему генерации случайного секрета, описанную в работе \cite{Gen}.

Предполагаем, что доверенный дилер случайным образом выбирает числа $r, r'$, публикует $Y = rG$ и затем делит $r$ между участниками процесса с помощью схемы Педерсена. Мы же хотим исключить из протокола дилера, сделать это можно следующим образом.\\

\textit{Каждый участник протокола $P_i$ действует так:}

\begin{enumerate}
    \item Каждый $P_i$ случайным образом выбирает $r_i, r'_i \in Z_q$ и действует как дилер в соответствии со схемой Педерсена. Пусть в качестве случайных многочленов выступают
    $$ f_i(u) = \sum_{t=0}^{k-1}a_{it}u^t , f'_i(u) = \sum_{t=0}^{k-1}a'_{it}u^t,$$ где $a_{i0} = r_i, a'_{i0} = r'_i$, тогда опубликованные величины примут значения $C_{it} = a_{it}G + a'_{it}H$ для $t\in\{0, ...,k-1\}$
    \item Пусть $H_0$ - подмножество игроков не уличённых в жульничестве на шаге 1. Тогда величина $r$, распространяемая в качестве секрета не может быть вычислена ниодним из участников, но при этом она равна $\sum_{i \in H_0}r_i$. Каждый $P_i$ вычисляет значение своей части секрета $s_i = \sum_{j \in H_0}f_j(i)$ mod q и величину $s'_i = \sum_{j \in H_0}f'_j(i)$ mod q.
    \item Представляем Y в виде $Y = \sum_{j \in H_0}r_jG$. Каждый участник из $H_0$ проверяет $Y_i = s_iG$ в соответсвии со схемой Фелдмана \cite{Fel}:
    \begin{enumerate}[label=\alph*)]
        \item Каждый участник $P_i$ для $i \in H_0$ вскрывает $A_{it} = a_{it}G$ для $t \in \{0, ...,k-1\}$. 
        \item Каждый $P_j$ проверяет значения вскрытые другими участниками из $H_0$. Точнее, для каждого $Pi$, $i \in H_0$ $P_j$ проверяет выполняется ли равенство
        \begin{equation}\label{second}
            f_i(j)G = \sum_{k=0}^{t-1} j^k A_{ik}.
        \end{equation}
        Если проверка провалилась для индекса $i$, $P_j$ подаёт жалобу на участника $P_i$ путём предъявления величин $f_i(j), f'_i(j)$, удовлетворяющих (\ref{check}), но не удовлетворяющих (\ref{second}).
        \item Для участников  $P_i$, кто получил как минимум одну обоснованную жалобу, например, значения удовлетворяющие (\ref{check}), но не удовлетворяющие (\ref{second}), остальные участники производят реконструкцию фазы схемы Педерсена, чтобы посчитать $r_i, f_i, A_{it}$ для $t \in \{0,...,k-1\}$. Т.е. каждый участник раскрывает своё значение $r_i$ и выбирает новое, удовлетворяющее (\ref{check}). Все участиники из $H_0$ полагают $Y_i = r_iG$. 
    \end{enumerate}
    
\end{enumerate}


\section{Подпись Шнорра}

какой-то текст...

\chapter{Пороговая подпись Шнорра}
\section{Реализация Стинсона и Стробла}
\subsection{Протокол}
\subsection{Надёжность}
\section{FROST}
\subsection{Протокол}
\subsection{Надёжность}
\chapter{Сравнение существующих реализаций}
\section{Сравнение реализации}
\subsection{Автономность участников}
\subsection{Количество раундов генерации ключа}
\section{Сравнение надёжности}