\chapter{Подпись Шнорра}

Здесь и далее мы используем постановку задачи дискретного логарифмирования на эллиптической кривой. Будем считать \textit{q} достаточно большим простым числом, а \textit{G} и \textit{H} генераторами подгрупп порядка \textit{q} на эллиптической кривой \textit{E}. Мы предполагаем, что кривая \textit{E} выбрана так, что проблема дискретного логарифмирования вычислительно трудна в подгруппе, порождённой \textit{G}, то есть невозможно за разумное время вычислить такое \textit{d}, что $G = dH$.  

\section{Схема разделения секрета Шамира}

В работе \cite{KG} Шамир ставит перед собой задачу поделить секрет \textit{s} на \textit{n} частей так, чтобы 
\begin{itemize}
    \item знание любых \textit{k} или более частей $s_i$ позволяло легко вычислить s;
    \item знание любых $k-1$ или менее частей $s_i$ оставляло \textit{s} неопределённым в том смысле, что все его возможные значения равновероятны.
\end{itemize}

Полученный алгоритм называется (\textit{k}, \textit{n}) пороговой схемой. Полученная схема базируется на интерполяционных многочленах Лагранжа и выглядит она следующим образом:

Пусть $s \in Z_q$, $n < q$, тогда дилер выбирает случайный многочлен $f$ степени не более, чем $k-1$ на полем $Z_q$, такой, что $f(0) = s.$. Тогда каждый игрок в качестве своего кусочка получает $s(i) = f(i)$.  

Существует ровно один многочлен степени не более, чем $k - 1$, удовлетворяющий $f(i) = s_i$ для \textit{k} значений \textit{i}. Таким образом любая коалиция $\mathcal{P}$ из \textit{k} человек может восстановить многочлен $f$ путём интерполяции Лагранжа:

$$ f(u) = \sum_{i \in \mathcal{P}}f(i)\omega_i(u), \text{ где } \omega_i(u) = \prod_{\substack{i \in P \\ j \neq i}} \frac{u - j}{i - j} \text{ mod q.} $$

Так как $s = f(0)$, коалиция $\mathcal{P}$ может восстановить секрет следующим образом:

$$ s = f(0) = \sum_{i\in\mathcal{P}}f(i)\omega_i, \text{ где } \omega_i = \omega_i(0) = \prod_{\substack{i \in P \\ j \neq i}} \frac{j}{j - i} \text{ mod q.}$$

Каждое $\omega_i \neq 0$ и может быть легко вычислено из открытой информации. $k - 1$ участника для восстановления не хватит, так как иначе им не удастся ннайти свободный член многочлена $f$, потому что он с равной вероятностью может принимать любое из своих значений.

\section{Проверяемое разделение секрета Педерсена}

Схема проверяемого разделения секрета нужна, чтобы предотвратить жульничество со стороны дилера. В этой схеме каждый участник может проверить свою часть секрета. Если дилер поделился неверным секретом, это будет обнаружено. В работе \cite{VSS} Педерсен представил свой варинант подобной схемы. Выглядит она следующим образом.

Предположим, что дилер имеет секрет $s \in Z_q$ и случайное число $s' \in Z_q$, и определяется парой чисел (\textit{s}, \textit{s'}) через открытую информацию $C_0 = sG + s'H.$ Тогда секрет \textit{s} может быть разделён между пользователями так:

\begin{enumerate}
    \item Дилер выбирает случайные многочлены
    $$ f(u) = s + f_1u + ...+f_{t-1}u^{t-1} \text{ и } f'(u) = s' + f'_1u + ... + f'_{t-1}u^{t-1}$$
    где $s, s', f_t, f'_t \in Z_q$ для $t \in \{1, ... , k-1\}$. Подсчитывает ($s_i, s'_i$) = ($f(i), f'(i)$) для $i \in \{1, ..., n\}$.

    \item Диллер секретно рассылает участникам пары $(s_i, s'_i)$.

    \item Дилер предъявляет значения $C_t = f_tG + f'_tH$ for $t \in \{1, ..., k-1\}$.
\end{enumerate}

\begin{enumerate}
    \item Каждый участник проверяет, что 
    \begin{equation}\label{check}
        s_iG + s'_iH = \sum_{t=0}^{k-1}i^tC_t. 
    \end{equation}
     Если проверка провалилась, участник подаёт жалобу на дилера.
     \item От каждой жалобы от \textit{i}-го участника дилер может защитить себя предъявив значения $f(i), f'(i)$, удовлетворяющие (\ref{check}).
     \item Отказываемся от дилера, если
     \begin{itemize}
         \item На шаге 1 он получил хотя бы \textit{k} жалоб;
         \item На шаге 2 он ответил на жалобу значениями, не удовлетворяющими (\ref{check}).
     \end{itemize}

     Опираясь на вычислительную трудность проблемы дискретного логарифма на эллиптическй кривой, Педерсен доказал, что любая коалифция из менее, чем k участников не сможет получить никакой информации о секрете.
\end{enumerate}

\section{Генерация случайного секрета для криптосистем, опирающихся на проблему дискретного логарифма}

Приведём схему генерации случайного секрета, описанную в работе \cite{Gen}.

Предполагаем, что доверенный дилер случайным образом выбирает числа $r, r'$, публикует $Y = rG$ и затем делит $r$ между участниками процесса с помощью схемы Педерсена. Мы же хотим исключить из протокола дилера, сделать это можно следующим образом.\\

\textit{Каждый участник протокола $P_i$ действует так:}

\begin{enumerate}
    \item Каждый $P_i$ случайным образом выбирает $r_i, r'_i \in Z_q$ и действует как дилер в соответствии со схемой Педерсена. Пусть в качестве случайных многочленов выступают
    $$ f_i(u) = \sum_{t=0}^{k-1}a_{it}u^t , f'_i(u) = \sum_{t=0}^{k-1}a'_{it}u^t,$$ где $a_{i0} = r_i, a'_{i0} = r'_i$, тогда опубликованные величины примут значения $C_{it} = a_{it}G + a'_{it}H$ для $t\in\{0, ...,k-1\}$
    \item Пусть $H_0$ - подмножество игроков не уличённых в жульничестве на шаге 1. Тогда величина $r$, распространяемая в качестве секрета не может быть вычислена ниодним из участников, но при этом она равна $\sum_{i \in H_0}r_i$. Каждый $P_i$ вычисляет значение своей части секрета $s_i = \sum_{j \in H_0}f_j(i)$ mod q и величину $s'_i = \sum_{j \in H_0}f'_j(i)$ mod q.
    \item Представляем Y в виде $Y = \sum_{j \in H_0}r_jG$. Каждый участник из $H_0$ проверяет $Y_i = s_iG$ в соответсвии со схемой Фелдмана \cite{Fel}:
    \begin{enumerate}[label=\alph*)]
        \item Каждый участник $P_i$ для $i \in H_0$ вскрывает $A_{it} = a_{it}G$ для $t \in \{0, ...,k-1\}$. 
        \item Каждый $P_j$ проверяет значения вскрытые другими участниками из $H_0$. Точнее, для каждого $Pi$, $i \in H_0$ $P_j$ проверяет выполняется ли равенство
        \begin{equation}\label{second}
            f_i(j)G = \sum_{k=0}^{t-1} j^k A_{ik}.
        \end{equation}
        Если проверка провалилась для индекса $i$, $P_j$ подаёт жалобу на участника $P_i$ путём предъявления величин $f_i(j), f'_i(j)$, удовлетворяющих (\ref{check}), но не удовлетворяющих (\ref{second}).
        \item Для участников  $P_i$, кто получил как минимум одну обоснованную жалобу, например, значения удовлетворяющие (\ref{check}), но не удовлетворяющие (\ref{second}), остальные участники производят реконструкцию фазы схемы Педерсена, чтобы посчитать $r_i, f_i, A_{it}$ для $t \in \{0,...,k-1\}$. Т.е. каждый участник раскрывает своё значение $r_i$ и выбирает новое, удовлетворяющее (\ref{check}). Все участиники из $H_0$ полагают $Y_i = r_iG$. 
    \end{enumerate}
    
\end{enumerate}

\section{Подпись Шнорра}

В работе \cite{Schnorr} Шнорр представил следующий вариант выработки подписи. Пусть ($x, Y$) - ключевая пара пользователя, \textit{m} - сообщение, $h(\cdot)$ - необратимая хэш-функция, а $G$ - точка эллиптической кривой, порождающая подгруппу порядка $q$. Тогда пользователь генерирует подпись Шнорра сообщения $m$ в соответствии со следующим алгоритмом:

\begin{enumerate}
    \item Случайным образом выбирает $e \in Z_q$
    \item Вычисляет $V = eG$
    \item Вычисляет $\sigma = e + h(m||V)x$ mod q
    \item Предполагает, что подписью сообщения $m$ будет ($V, \sigma$)
\end{enumerate}

Пара ($V, \sigma$) действительно будет подписью сообщения $m$, тогда и только тогда, когда $\sigma \in Z_q$ и $$ \sigma G = V + h(m||V)Y. $$

Также в \cite{Schnorr} Шнорр, используя лемму о разветвлении, показывает, что задача <<единичная подделка>> в схеме ROM по сложности сопоставима с задачей дискретного логарифмирования в подгруппе, порождённой $G$. 
  
\chapter{Пороговая подпись Шнорра}

Схема подписи Шнорра широко используется в современном мире, в том числе в белорусских стандартах. Она хороша тем, что легко преобразуется в пороговую, то есть любые $t$ из $n$ участников смогут вырабатывать подпись, используя личные части ключа. Мы же рассмотрим две важные реализации и в последствии сравним их. 

\section{Реализация Стинсона и Стробла}

Реализация Стинсона и Стробла \cite{SS} является классическим примером надёжной схемы, т.е. если все $t$ участников исправно следуют протоколу, то протокол гарантированно завершится без ошибок, даже если подмножество из не более, чем $n-t$ участников скомпроментировано. 

Данная конструкция требует не менее 4 тактов для совершения подписи (учитывая, что все участники ведут себя хорошо): три такта для генерации ключа (см. 1.3) и один такт чтобы раздать секреты и выработать групповую подпись. Каждый такт обязует всех участников обменяться данными со всеми.

\subsection{Протокол}

Протокол состоит из двух частей: генерации ключа и выработки подписи. Пусть $P_1, P_2, ... , P_n$ множество подписантов, $G$ - генератор подгруппы порядка $q$ точек эллиптичекой кривой.

\subsubsection{Протокол генерации ключа}

Все $n$ подписантов должны скооперироваться для генерации открытого ключа, далее секретный ключ делится между всеми $P_j$. Они генерируют случайный секрет в соответсвии с протоколом из пункта 1.3. Пусть на выходе получилось:

$$ (\alpha_1, ... , \alpha_n) \xleftrightarrow{(t, n)} (x|Y, b_kG, H_0), k\in \{1, ...,t-1\}.$$

Для любого $j \in H_0$, $\alpha_j$ - часть секрета, принадлежащая участнику $P_j$, которая в дальнейшем будет использоваться для выработки ключевой пары $(x, Y)$, где $x$ - личный ключ, а $Y$ - открытый.

\subsubsection{Протокол выработки подписи}

Пусть $m$ - сообщение, а $h(\cdot)$ - необратимая хэш-функция. Предположим, что  $H_1 \subseteq H_0$ множество индексов участников, которые хотят выработать подпись. Тогда они должны следовать следующему протоколу:

\begin{enumerate}
    \item Если $|H_1| < t$, стоп. В противном случае, подмножество $H_1$ генерируют случайный секрет по схеме из пункта 1.3. Пусть на выходе получилось:

    $$ (\beta_1, ... , \beta_n) \xleftrightarrow{(t, n)} (e|V, c_kG, H_2), k\in \{1, ...,t-1\}.$$

    \item Если $|H_2| < t$, стоп. В противном случае, каждый $P_i$, $i \in H_2$ раскрывает

    $$ \gamma_i = \beta_i + h(m||V)\alpha_i .$$

    \item Каждый $P_i$, $i \in H_2$ проверяет, что

    \begin{equation}\label{verify}
        \gamma_jG = V + \sum_{k=1}^{t-1}c_kj^kG + h(m||V) \Big(Y + \sum_{k=1}^{t-1}b_kj^kG \Big) \text{ } \forall j\in H_2.
    \end{equation}

    Пусть $H_3$ - множество индексов участников не уличённых в жульничестве на шаге 3.

    \item Если $|H_3| < t$, стоп. В противном случае, каждый $P_i$, $i \in H_3$ выбирает произвольное подмножество $H_4 \subseteq H_3$ с $|H_4| = t$ и вычисляет $\sigma$, удовлетворяющее $\sigma = e + h(m||V)x$, где

    $$ \sigma = \sum_{j \in H_4}\gamma_j\omega_j \text{ и } \omega_j = \prod_{\substack{l \in H_4 \\ l \neq j}}\frac{l}{l - j}.$$

    Подпись это $(\sigma, V)$. Подпись может быть проверена так же, как и в оригинальной схеме Шнорра: $ \sigma G = V + h(m||V)Y \text{ и } \sigma\in Z_q .$
\end{enumerate}

\subsection{Корректность}

Мы хотим показать, что подпись $\sigma$, выработанная на шаге 4 - это на самом деле подпись Шнорра $m$, другими словами, $\sigma G = e + h(m||V)x \text{ } mod \text{ } q$. Пусть $F_1$ - случайный полином из протокола генерации ключа $(\alpha_i = F_1(i), i \in H_0)$, а $F_2$ - многочлен полученный на шаге 1 ($\beta_i = F_2(i), i\in H_1$). Более того, пусть $F_3 := F_2 + h(m||V)F_1$. Поскольку $\gamma_i = F_3(i), i\in H_3$, из формулы интерполяции Лагранжа следует, что участники вычисляют $\sigma = F_3(0)$. Мы можем рассуждать следующим образом:

$$ \sigma = F_3(0) = F_2(0) + h(m||V)F_1(0) = e + h(m||V)x. $$

\subsection{Надёжность}

Мы хотим показать, что, если меньше, чем $t$ подписантов скомпрроментирваны, схема всегда вырабатывает корректную подпись. Мы предполагаем, что $t \leq \frac{n}{2}$. 

Из надёжности протокола генерации секрета следует, что каждый честный участник $P_i$ вычисляет корректные $\alpha_i, \beta_i, \gamma_i$. Тогда остаётся хотя бы $t$ честных игроков, которые могут корректно проверить $\gamma_i$, подставив его в \ref{verify}, это прямиком следует из того, что честные игроки всегда осуществляют проверку корректно.  

\subsection{Безопасность}

Пороговая подпись Шнорра безопасна, так как задача "единичная подделка" в схеме ROM для неё сапоставима по сложности с задачей дискретного логарифмирования в подгруппе, порождённой $G$. Для этого в \cite{SS} вводится противник $A_{DistSchnorr}$ и доказывается, что если он умеет атаковать пороговую подпись, то он умеет атаковать и обычную подпись Шнорра, и наоборот. Для этого используется симулятор SIM, который играет роль честного подписанта. 

\begin{defenition}
    Предположим, что $H_0$ - подмножество участников вырабатывающих секрет для входных данных ($q, G$) и получающих на выходе $Y$. Пусть $\tilde{A}$ - противник, который может скомпроментировать не более $t-1$ подписанта. Пусть $view(\tilde{A}, G, q, Y)$ определяет кругозор противника для этого протокола. Пусть $VIEW(\tilde{A}, G, q, Y)$ - случайная величина порождённая $view(\tilde{A}, G, q, Y)$.
\end{defenition}

\begin{lemma}
    Для любого полиномиального вероятностного алгоритма существует противник $\tilde{A}$ и полиномиальный симулятор SIM, который может посчитать случайную величину $SIM(G, q, Y)$,  
\end{lemma}

\begin{defenition}
    Пусть $A_{NormSchnorr}$ вероятностный алгоритм, соответсвующий противнику, который может спросить подписанта о правильной подписи. Назовём $A_{NormSchnorr}(G, q, Y)$ случайную величину, отвечающую вероятности события, что $A_{NormSchnorr}$ задаёт вопросы ($m_1, m_2, ...$) и получает в ответ ($\tilde{m}, \tilde{\sigma}, \tilde{V}$) (при исходных данных ($G, q, Y$)). Вероятность вычисляется по всем возможным ипостасям $A_{NormSchnorr}$ и подписантов. 
\end{defenition}

\begin{defenition}
    Пусть $A_{DistSchnorr}$ полиномиальный вероятностный алгоритм противника, который может скомпроментировать до $t-1$ подписанта. Он также может иметь $t$ или больше произвольных участников вырабатывающих подпись по его запросу. Обозначим за $A_{DistSchnorr}(G, q|Y)$ случайную величину, отвечающую вероятности события, что $A_{NormSchnorr}$ задаёт вопросы ($m_1, m_2, ...$) (при исходных данных ($G, q$) и в конце получает ($\tilde{m}, \tilde{\sigma}, \tilde{V}$) при условии, что протокол выработки ключа вернул $Y$. 
\end{defenition}

\begin{theorem}
    Для любого противника $A_{NormSchnorr}$ против $D_{NormSchnorr}$ (обозначает схему Шнорра), существует противник $A_{DistSchnorr}$ против $D_{DistSchnorr}$ (обозначает пороговую схему Шнорра), такой что

    $$ Pr[A_{DistSchnorr}(G, q|Y) = (m_1, ...,(\tilde{m}, \tilde{\sigma}, \tilde{V}))] = $$
    $$Pr[A_{NormSchnorr}(G, q, Y) = (m_1, ...,(\tilde{m}, \tilde{\sigma}, \tilde{V}))] $$
\end{theorem}

\begin{theorem}
    Для любого противника $A_{DistSchnorr}$ против $D_{DistSchnorr}$ (обозначает схему Шнорра), существует противник $A_{NormSchnorr}$ против $D_{NormSchnorr}$ (обозначает пороговую схему Шнорра), такой что

    $$ Pr[A_{DistSchnorr}(G, q, Y) = (m_1, ...,(\tilde{m}, \tilde{\sigma}, \tilde{V}))] = $$
    $$Pr[A_{NormSchnorr}(G, q|Y) = (m_1, ...,(\tilde{m}, \tilde{\sigma}, \tilde{V}))] $$
\end{theorem}

\section{FROST}
\subsection{Протокол}
\subsection{Надёжность}
\chapter{Сравнение существующих реализаций}
\section{Сравнение реализации}
\subsection{Автономность участников}
\subsection{Количество раундов генерации ключа}
\section{Сравнение надёжности}