\chapter*{ \large ЗАКЛЮЧЕНИЕ}
\addcontentsline{toc}{chapter}{ЗАКЛЮЧЕНИЕ}

По результатам проделланной работы можно сделать вывод, что обе реализации имеют свои недостатки, однако это не отменяет их эффективности. FROST конечно более современен и работает быстрее в идеальных условиях, однако такая реализация из-за возможности параллельного выполнения процессов подвержена атаке Дрейверса. Реализация Стинсона и Стробла данной атаке не подвержена, но требует большего количества ресурсов времени и памяти.

Даже рассмотренная схема FROST не является последней модификацией. Работа над ЭЦП активно продолжается, однако на повестке дня вопрос о парольном усилении пороговой подписи, так как давно актуален вопрос внедрения ЭЦП на ПК и смартфоны. К сожалению, исследований в этом направлении пока немного, но мы надеемся, что вскоре парольное усиление будет разработано и стандартизировано.