\chapter*{\large ВВЕДЕНИЕ}  
\addcontentsline{toc}{chapter}{ВВЕДЕНИЕ}
Электронные цифровые подписи уже стали неотъемлемой частью современного документооборота, особенно активно они применяются при совершении сделок связанных с криптовалютами, в связи с этим встал вопрос о наделении ЭЦП пороговыми свойствами, то есть возможностью создания подписи для документа любыми $k$ или более из $n$ участниками коалиции, но никие группы из $k-1$ или менее участников уже не смогут выработать подпись, кроме того, теперь каждый участник коалиции будет иметь свою часть секретного ключа, известную только ему.

В рамках данной работы была рассмотрена сама подпись Шнорра и, сопутствующие ей схема разделения секрета Шамира и протокол генерации случайного ключа Педерсена. Реализаций пороговой схемы существует много, но было рассмотрено две классические модели: схема Стинсона и Стробла и FROST. Также проведено их последующее сравнение, в частности как надёжной и ненадёжной схем. 